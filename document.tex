%------------------------------------------------------------------------------
% Template file for the submission of papers to IUCr journals in LaTeX2e
% using the iucr document class
% Copyright 1999-2003 International Union of Crystallography
% Version 1.2 (11 December 2002)
%------------------------------------------------------------------------------

\documentclass{iucr}              % DO NOT DELETE THIS LINE

                   %%%%%%%%%%%%%%%%%%%%%%%%%%%%%%%%%%%%%%%%
                   \def\href#1{\relax}\let\foo\caption
%%%%%%%%%%%%%%%%%%%%%%%%%%%%%%%%%%%%%%%%%%%%%%%%%%%%%%%%%%%
% To generate hyperlinked PDF, change the \documentclass line
% to say \documentclass[pdf]{iucr} and process with pdflatex
\ifPDF
  \RequirePackage{hyperref}
  \PassOptionsToPackage{pdftex,bookmarksopen,bookmarksnumbered}{hyperref}
  \voffset=-0.5in
\fi
\let\caption\foo
%%%%%%%%%%%%%%%%%%%%%%%%%%%%%%%%%%%%%%%%%%%%%%%%%%%%%%%%%%%

%------------------------------------------------------------------------------
% Information about the type of paper
%------------------------------------------------------------------------------
     \paperprodcode{a000000}      % Replace with production code if known
     \paperref{xx9999}            % Replace xx9999 with reference code if known
     \papertype{IU}               % Indicate type of article
                                  %   FA - research papers (full article)
                                  %   SC - short communications
                                  %   FC - fast communications
                                  %   LA - lead article
                                  %   TR - topical review
                                  %   XL - crystallization papers
                                  % (Following categories rarely in LaTeX)
                                  %   AA - abstracts
                                  %   AD - addenda and errata
                                  %   AI - inorganic compounds
                                  %   AM - metal-organic compounds
                                  %   AO - organic compounds
                                  %   BC - books received
                                  %   BR - book reviews
                                  %   BI - biography
                                  %   CA - cif applications
                                  %   CD - crystal data
                                  %   CE - current events
                                  %   CI - inorganic compounds
                                  %   CL - calendar of events
                                  %   CM - metal-organic compounds
                                  %   CN - cryocrystallography papers
                                  %   CO - organic compounds
                                  %   CP - computer programs
                                  %   CR - crystallographers
                                  %   CS - scientific comment
                                  %   ED - editorial
                                  %   EI - inorganic compounds
                                  %   EM - metal-organic compounds
                                  %   EO - organic compounds
                                  %   FI - inorganic compounds
                                  %   FM - metal-organic compounds
                                  %   FO - organic compounds
                                  %   IP - issue preface
                                  %   IU - iucr
                                  %   LE - letters to the editor
                                  %   LN - laboratory notes
                                  %   ME - forthcoming meetings/short courses
                                  %   MR - meeting reports
                                  %   NN - notes and news
                                  %   NP - new commercial products
                                  %   OB - obituaries
                                  %   PA - computer program abstracts
                                  %   RI - reference information
                                  %   SG - structural genomics papers
                                  %   SI - short format inorganic compounds
                                  %   SM - short format metal-organic compounds
                                  %   SO - short format organic compounds
                                  %   SP - short structural papers
                                  %   SR - software reviews
                                  %   TE - teaching and education

     \paperlang{english}          % Can be english, french, german or russian
%------------------------------------------------------------------------------
% Information about journal to which submitted
%------------------------------------------------------------------------------
     %\journalcode{A}             % Indicate the journal to which submitted
                                  %   A - Acta Crystallographica Section A
                                  %   B - Acta Crystallographica Section B
                                  %   C - Acta Crystallographica Section C
                                  %   D - Acta Crystallographica Section D
                                  %   J - Journal of Applied Crystallography
                                  %   S - Journal of Synchrotron Radiation
     %-------------------------------------------------------------------------
     % The following entries will be changed as required by editorial staff
     %-------------------------------------------------------------------------
     \journalyr{2017}
     %\journaliss{1}
     %\journalvol{57}
     %\journalfirstpage{1}
     %\journallastpage{11}
     \journalreceived{\relax}
     \journalaccepted{\relax}
     \journalonline{\relax}

\begin{document}                  % DO NOT DELETE THIS LINE

     %-------------------------------------------------------------------------
     % The introductory (header) part of the paper
     %-------------------------------------------------------------------------

     % The title of the paper. Use \shorttitle to indicate an abbreviated title
     % for use in running heads (you will need to uncomment it).

\title{Calibration of the experimetal setup in pyFAI with a goniometer
mounted detector}
\shorttitle{Using pyFAI with a goniometer}

     % Authors' names and addresses. Use \cauthor for the main (contact) author.
     % Use \author for all other authors. Use \aff for authors' affiliations.
     % Use lower-case letters in square brackets to link authors to their
     % affiliations; if there is only one affiliation address, remove the [a].

     \author[a]{J.}{Kieffer}
     \author[a]{V.}{Valls}
     \cauthor[a]{jerome}{kieffer@esrf.fr}{}
     %\aufn{On leave from Institute
     %of Advanced Research, Albany, Ruritania.}

     \aff[a]{ESRF \city{Grenoble}, \country{France}}
     %\aff[b]{3 Watery Way, \city{Full Fathom} 5, \country{Atlantis}}

     % Use \shortauthor to indicate an abbreviated author list for use in
     % running heads (you will need to uncomment it).

\shortauthor{Kieffer et al.}

     % Use \vita if required to give biographical details (for authors of
     % invited review papers only). Uncomment it.


%\vita{Joe Soape is an archetypal generic author, whose association with the
%much-travelled Kilroy has extended over many years. He travels to work each
%day on a Clapham omnibus.
%\\
%John Doe is also a generic individual with extensive experience of legal and
%forensic matters.}

     % Keywords (required for Journal of Synchrotron Radiation only)
     % Use the \keyword macro for each word or phrase, e.g.
     % \keyword{X-ray diffraction}\keyword{muscle}

\keyword{\LaTeX}
\keyword{class file}
\keyword{documentation}

     % PDB and NDB reference codes for structures referenced in the article and
     % deposited with the Protein Data Bank and Nucleic Acids Database (Acta
     % Crystallographica Section D). Repeat for each separate structuree.g.
     % \PDBref[dethiobiotin synthetase]{1byi} \NDBref[d(G$_4$CGC$_4$)]{ad0002}

%\PDBreference[optional name]{refcode}
%\NDBreference[optional name]{refcode}

\maketitle                        % DO NOT DELETE THIS LINE

\begin{synopsis}
Explain how to obtain a powder diffraction pattern from
many images images taken with a 2D X-Ray detector mounted on a moving arm
(goniometer) and how to define precisely the position of the detector on the
moving arm from Debye-Scherrer ring of a reference compound.
\end{synopsis}

\begin{abstract}
TODO
\end{abstract}


     %-------------------------------------------------------------------------
     % The main body of the paper
     %-------------------------------------------------------------------------
     % Now enter the text of the document in multiple \section's, \subsection's
     % and \subsubsection's as required.


\section{Introduction}

Area detectors mounted on goniometer arms are commonly available for
powder-diffraction in lab-source diffractometer (for example Rigaku
HyPix-3000).
%The larger number of pixels trades-of the resolution for speed.
On the oposite this moving detector setup is rarely used at synchrotrons where
larger detectors are often prefered and kept fix during the whole acquisition.
The fixed position setup combined with the speed of modern detectors allows
easily the acquisition in kinetic mode for following chemical reaction or other
physical processes.

Nevertheless high-Q acquisition is needed for Pair-wise Distribution Function
(PDF) analysis which requires even larger detectors and higher energies to be
able to cover the Q-range with one single frame.
When speed is not a critical parameter, such experiment could be done with a
smaller detector mounted on a moving arm; this setup being commonly available
on most diffraction beamlines.

This document presents first the pyFAI library then how to merge multiple
diffraction images acquired at different positions with this library.
Then it explains how to calibrate the absolute position of every single
pixel in the detector as function of the parameter of the goniometer or of the
translation table.

\section{The fast azimuthal integration library}

PyFAI is a Python library used to tranform 2D images into 1D powder diffraction
pattern using rebinning of the pixel position to polar coordinates.
It provides in addition tools to calibrate the detector position, i.e. determine
where it is located in space from the ellipses drawn by the Debye-Scherrer cones
intersected

\section{Azimuthal integration of multiple frames taken at multiple geometries}



\section{Style of the paper}


\section{The header of the paper}


\end{document}
