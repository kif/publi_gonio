%------------------------------------------------------------------------------
% Template file for the submission of papers to IUCr journals in LaTeX2e
% using the iucr document class
% Copyright 1999-2003 International Union of Crystallography
% Version 1.2 (11 December 2002)
%------------------------------------------------------------------------------

\documentclass{iucr}              % DO NOT DELETE THIS LINE

                   %%%%%%%%%%%%%%%%%%%%%%%%%%%%%%%%%%%%%%%%
                   \def\href#1{\relax}\let\foo\caption
%%%%%%%%%%%%%%%%%%%%%%%%%%%%%%%%%%%%%%%%%%%%%%%%%%%%%%%%%%%
% To generate hyperlinked PDF, change the \documentclass line
% to say \documentclass[pdf]{iucr} and process with pdflatex
\ifPDF
  \RequirePackage{hyperref}
  \PassOptionsToPackage{pdftex,bookmarksopen,bookmarksnumbered}{hyperref}
  \voffset=-0.5in
\fi
\let\caption\foo
%%%%%%%%%%%%%%%%%%%%%%%%%%%%%%%%%%%%%%%%%%%%%%%%%%%%%%%%%%%

%------------------------------------------------------------------------------
% Information about the type of paper
%------------------------------------------------------------------------------
     \paperprodcode{a000000}      % Replace with production code if known
     \paperref{xx9999}            % Replace xx9999 with reference code if known
     \papertype{IU}               % Indicate type of article
                                  %   FA - research papers (full article)
                                  %   SC - short communications
                                  %   FC - fast communications
                                  %   LA - lead article
                                  %   TR - topical review
                                  %   XL - crystallization papers
                                  % (Following categories rarely in LaTeX)
                                  %   AA - abstracts
                                  %   AD - addenda and errata
                                  %   AI - inorganic compounds
                                  %   AM - metal-organic compounds
                                  %   AO - organic compounds
                                  %   BC - books received
                                  %   BR - book reviews
                                  %   BI - biography
                                  %   CA - cif applications
                                  %   CD - crystal data
                                  %   CE - current events
                                  %   CI - inorganic compounds
                                  %   CL - calendar of events
                                  %   CM - metal-organic compounds
                                  %   CN - cryocrystallography papers
                                  %   CO - organic compounds
                                  %   CP - computer programs
                                  %   CR - crystallographers
                                  %   CS - scientific comment
                                  %   ED - editorial
                                  %   EI - inorganic compounds
                                  %   EM - metal-organic compounds
                                  %   EO - organic compounds
                                  %   FI - inorganic compounds
                                  %   FM - metal-organic compounds
                                  %   FO - organic compounds
                                  %   IP - issue preface
                                  %   IU - iucr
                                  %   LE - letters to the editor
                                  %   LN - laboratory notes
                                  %   ME - forthcoming meetings/short courses
                                  %   MR - meeting reports
                                  %   NN - notes and news
                                  %   NP - new commercial products
                                  %   OB - obituaries
                                  %   PA - computer program abstracts
                                  %   RI - reference information
                                  %   SG - structural genomics papers
                                  %   SI - short format inorganic compounds
                                  %   SM - short format metal-organic compounds
                                  %   SO - short format organic compounds
                                  %   SP - short structural papers
                                  %   SR - software reviews
                                  %   TE - teaching and education

     \paperlang{english}          % Can be english, french, german or russian
%------------------------------------------------------------------------------
% Information about journal to which submitted
%------------------------------------------------------------------------------
     %\journalcode{A}             % Indicate the journal to which submitted
                                  %   A - Acta Crystallographica Section A
                                  %   B - Acta Crystallographica Section B
                                  %   C - Acta Crystallographica Section C
                                  %   D - Acta Crystallographica Section D
                                  %   J - Journal of Applied Crystallography
                                  %   S - Journal of Synchrotron Radiation
     %-------------------------------------------------------------------------
     % The following entries will be changed as required by editorial staff
     %-------------------------------------------------------------------------
     \journalyr{2017}
     %\journaliss{1}
     %\journalvol{57}
     %\journalfirstpage{1}
     %\journallastpage{11}
     \journalreceived{\relax}
     \journalaccepted{\relax}
     \journalonline{\relax}

\begin{document}                  % DO NOT DELETE THIS LINE

     %-------------------------------------------------------------------------
     % The introductory (header) part of the paper
     %-------------------------------------------------------------------------

     % The title of the paper. Use \shorttitle to indicate an abbreviated title
     % for use in running heads (you will need to uncomment it).

\title{Calibration of the experimetal setup in pyFAI with a goniometer
mounted detector}
\shorttitle{Using pyFAI with a goniometer}

     % Authors' names and addresses. Use \cauthor for the main (contact) author.
     % Use \author for all other authors. Use \aff for authors' affiliations.
     % Use lower-case letters in square brackets to link authors to their
     % affiliations; if there is only one affiliation address, remove the [a].

     \cauthor[a]{J.}{Kieffer}{jerome.kieffer@esrf.eu}
     \author[a]{J.}{Kieffer}
     \author[a]{V.}{Valls}
     
     %\aufn{On leave from Institute
     %of Advanced Research, Albany, Ruritania.}

     \aff[a]{ESRF \city{Grenoble}, \country{France}}
     %\aff[b]{3 Watery Way, \city{Full Fathom} 5, \country{Atlantis}}

     % Use \shortauthor to indicate an abbreviated author list for use in
     % running heads (you will need to uncomment it).

\shortauthor{Kieffer et al.}

     % Use \vita if required to give biographical details (for authors of
     % invited review papers only). Uncomment it.


%\vita{Joe Soape is an archetypal generic author, whose association with the
%much-travelled Kilroy has extended over many years. He travels to work each
%day on a Clapham omnibus.
%\\
%John Doe is also a generic individual with extensive experience of legal and
%forensic matters.}

     % Keywords (required for Journal of Synchrotron Radiation only)
     % Use the \keyword macro for each word or phrase, e.g.
     % \keyword{X-ray diffraction}\keyword{muscle}

\keyword{\LaTeX}
\keyword{class file}
\keyword{documentation}

     % PDB and NDB reference codes for structures referenced in the article and
     % deposited with the Protein Data Bank and Nucleic Acids Database (Acta
     % Crystallographica Section D). Repeat for each separate structuree.g.
     % \PDBref[dethiobiotin synthetase]{1byi} \NDBref[d(G$_4$CGC$_4$)]{ad0002}

%\PDBreference[optional name]{refcode}
%\NDBreference[optional name]{refcode}

\maketitle                        % DO NOT DELETE THIS LINE

\begin{synopsis}
Explain how to obtain a powder diffraction pattern from
many images images taken with a 2D X-Ray detector mounted on a moving arm
(goniometer) and how to define precisely the position of the detector on the
moving arm from Debye-Scherrer ring of a reference compound.
\end{synopsis}

\begin{abstract}
TODO
\end{abstract}


     %-------------------------------------------------------------------------
     % The main body of the paper
     %-------------------------------------------------------------------------
     % Now enter the text of the document in multiple \section's, \subsection's
     % and \subsubsection's as required.


\section{Introduction}

Area detectors mounted on goniometer arms are commonly available for
powder-diffraction data acquisition in lab-source diffractometer (for example
Rigaku HyPix-3000).
%The larger number of pixels trades-of the resolution for speed.
On the oposite, moving detector setups are rarely used at synchrotrons for
the data acquisition itself, even if most of the detectors are mounted on a
moving table.
So at large facilities like synchrotons, larger detectors are 
prefered and kept fix during the whole acquisition, often to ease the
data-reduction step.
The fixed position setup combined with the speed of modern detectors allows
easily the acquisition in kinetic mode for following chemical reaction or other
physical processes.

Nevertheless high-Q acquisition is needed for Pair-wise Distribution Function
(PDF) analysis which requires even larger detectors and higher energies to be
able to cover the Q-range with one single frame.
When speed is not a critical parameter, such experiment could be done with a
smaller detector mounted on a moving arm which is moved during the
acquisition; this setup being already available on most diffraction beamlines.

This document presents first shortly the pyFAI library\cite{} then how to merge
multiple diffraction images acquired at different positions with this
library\cite{}. 
Then it explains how to calibrate the absolute position of every single
pixel in the detector when mounted on a goniometer (or of the translation table)
as function of the parameter of the goniometer. 

\section{The fast azimuthal integration library}

PyFAI is a Python library used to tranform 2D diffraction images into 1D powder
diffraction pattern using rebinning of the pixel position to polar coordinates.
It provides in addition tools to calibrate the detector position, i.e. determine
where it is located in space from the conics drawn by the Debye-Scherrer cones
intersected by the detector (ellipses when the detector is planar and slightly
inclined). This document describes the processing implemented in pyFAI
v0.14 which is to be published in 2017/still under development.

Azimuthal integration is performed in two steps, the first is a pixel-wise
transformation corresponding the image correction:
$$
I_{cor} = \frac{signal}{normalization}  = \frac{I_{raw} - I_{dark}}{F \times
\Omega \times P \times A } $$
where $I_{raw}$ is the raw detector signal, $I_{dark}$ is the dark current
image (may be also the background image for certain experiments), $F$ referes to
the flat-field correction, $\Omega$ to the solid angle of the given pixel, $P$
referes to polarization correction and $A$ referes to detector efficiency due
to in volume absorption due to parallax effects.

For azimuthal integration the numerator of this formula, herafter refered as
\textit{signal}, is separeted from the denominator which is refered as
\textit{normalization}.

The rebinning of the data is performed using an histogram of the position ($q$
or $2\theta$ values) weighted by the \textit{signal}.
This gives the sum of all signals withing a ring.
A second histogram of the positions is calculated, weighted by the
\textit{normalization} this time, which gives the sum of all normalizations
values.

The average signal over a ring is then simply the ratio of the two histograms.

$$
<I>_{ring} = \frac{\sum\limits_{i \in ring} c_i \times signal_i}
                  {\sum\limits_{i \in ring} c_i \times normalization_i} 
$$

In this equation, the parameter $c_i$ correspond to the fraction of area of a
pixel falling into a specific bin. 
As pyFAI provides multiple pixel splitting schemes, the differ only by their
$c_i$ coefficients. 
Simple weighted histograms corresponds actually to no splitting at all; in this
case the parameter $c_i$ is 1 for pixels falling in the bin an 0 for all the
others.
  
Those multiple pixel splitting schemes have already been described in
\cite{fv5028} and can be stored to accelerate the calculation of the
histogram \cite{kieffer_ashiotis-proc-euroscipy-2014}.

\section{Azimuthal integration of multiple frames taken at multiple geometries}

The integration of multiple images taken at varying position has first been
reported for pyFAI in \cite{PyFAI_PDJ}. 
The procedure is conceptually similar to the integration on a single image,
except that the various histograms, all performed on the same grid, are summed
together, signals from all imges on the numerator and normalizations from all
image on the denominator.

$$
<I>_{ring} = \frac{\sum\limits_{imges} \sum\limits_{i \in ring} c_i \times
signal_i} {\sum\limits_{imges} \sum\limits_{i \in ring} c_i \times
normalization_i} 
$$

The normalization for solid angle correction $\Omega$ has to be performed in
absolute solid-angle (unlike in single frame integration) as different
geometries may have very different sample-detector distances. 
This explains why a single image integrated in multi-geometry mode  has
intensity orders of magnitude larger than with the normal pyFAI integration
methods.

\section{Calibration using a Debye-Scherrer diffraction image}

The calibration of a detector position is performed using the Debye-Scherrer
rings collected from a reference powder called \textit{calibrant}.
The rings are extracted automatically and control points are placed at the
local maxima on the rings.
The geometry of the experiment is obtained from a least squares fitting of
the $2\theta$ angles.
In this work we will call them ``rings'' even if, for planar detector,
they are actually the conic intersections of the X-ray beam cones
with the detector plane.
For non planar detectors detector (also supported by pyFAI) those ``rings'' can
be any type of curve.

The least squares refinement is performed on an internal paramter-set containing
6  parameters (or 7 with the wavelength), definined around the
concept of Point Of Normal Incidence (hereafter named \textsc{poni}) which is
the orthongonal projection  of the sample position on the detector plan 
(or z=0 when the detector is non-planar and z varies from pixel to pixel).
This \textsc{poni} differs from the beam-center used in programs like
Fit2D\cite{fit2d} as the \textsc{poni} is always defined and often lies within
the detector's image.
So the parameters refined are the following:
\begin{itemize}
  \item \textit{dist}: the distance in meter from the sample position to the
  \textsc{poni}
  \item \textit{poni1} and \textit{poni2}: coordinate of the
  \textsc{poni} in meter within the detector plan (z=0) along the slow and fast
  dimention of the detector image (usually the row number and the column
  number, i.e. ``y, x'').
  \item \textit{rot1}, \textit{rot2} and \textit{rot3} which correspond to the
  rotation, expressed in radians, of the detector placed at the proper
  distance of the sample, along the 3 axis of the laboratory. The detector is
  first rotated around the verical axis (rot1), then around the horizontal axis
  (rot2) and finally around the incomming beam (rot3). 
  \textsc{poni}
\end{itemize}

This parameter set allows the description of any detector position in space.
The drawback is that many parameter are correlated: 
\begin{itemize}
  \item \textit{dist-wavelenght}: reducing the wavelength is equivalent to
  increase the distance unless the diffraction angle $2\theta$ is actually
  large. It is adviced to fix one of the the two. 
  \item \textit{rot1-poni2} and \textit{rot2-poni1} as a small rotation can be
  interepreted as a larger translation. 
\end{itemize}
Those correlations between parameters can be addressed by fixing one of
the two parameters which provides a much better fit (numerical stability).

The calibration is currently performed using a semi-graphical
application called \textit{pyFAI-calib}, available since the origin of the library 
which generates a \textsc{poni}-file containing the geometry and a
\textsc{npt}-file containing the control points used to perform the
least-squares fitting.
A new graphical application, replacing the current one for performing this
fitting is under development and will be available for the final version of
pyFAI v0.14.

\section{Calibration of a goniometer}


\ack{Acknowledgements}
TODO
We would like to thank all ESRF beamline teams for supporting the
pyFAI development, especially BM01, BM02, ID02, ID11, ID13, ID15, ID21, ID23,
BM26, ID29, BM29 and ID30.
In the instrumentation division (ISDD) we would like to thank Claudio
Ferrero, head of data analysis unit, for the critical revision of this
manuscript and Andy G\"otz, head of software group, for supporting the
algorithmic work performed on pyFAI in addition to the features exposed to the
user.

\end{document}
